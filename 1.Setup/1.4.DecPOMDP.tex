\section{Модель взаимодействия агента со средой}

\subsection{Среда}

Как и в оптимальном управлении, для моделирования влияния агента на среду в вероятности переходов достаточно добавить зависимость от выбираемых агентом действий. Итак, наша модель среды --- это <<управляемая>> марковская цепь.

\begin{definition} 
\emph{Средой} (environment) называется тройка $(\St, \A, \Trans)$, где: 
\begin{itemize}
    \item $\St$ --- \emph{пространство состояний} (state space), некоторое множество.
    \item $\A$ --- \emph{пространство действий} (action space), некоторое множество.
    \item $\Trans$ --- \emph{функция переходов} (transition function) или \emph{динамика среды} (world dynamics): вероятности $p(s' \HM\mid s, a)$.
\end{itemize}
\end{definition}


